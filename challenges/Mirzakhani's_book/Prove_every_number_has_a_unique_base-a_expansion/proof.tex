\documentclass[12pt]{article}
\usepackage{amsmath,amssymb,amsthm}
\usepackage{geometry}
\geometry{a4paper,margin=1in}
\usepackage{setspace}
\onehalfspacing

\title{Exercise 8 – Representation of Natural Numbers in Base \emph{a}}
\author{From \textit{Number Theory} by Maryam Mirzakhani and Roya Beheshti}
\date{}

\begin{document}
\maketitle

\section*{Problem}

\noindent
Let \(a\) be a natural number greater than \(1\).
Prove that every natural number \(n\) can be written uniquely in the form
\[
n = c_0 + c_1 a + c_2 a^2 + \cdots + c_m a^m,
\]
where each coefficient satisfies \(0 \le c_i \le a-1\).
We denote this representation by
\[
n = (c_m c_{m-1} \ldots c_1 c_0)_a.
\]

\section*{Proof}

\textbf{Existence.}
We prove by induction on \(n\) that such a representation exists.

\smallskip
If \(0 \le n < a\), then we can take \(m = 0\) and \(c_0 = n\);
thus the claim holds for all \(n < a\).

\smallskip
Assume now that the statement holds for all numbers smaller than some \(n \ge a\),
and consider \(n\) itself.
By the division algorithm, there exist unique integers \(q,r\) such that
\[
n = aq + r, \qquad 0 \le r < a.
\]
Here \(r\) will serve as the least significant digit, that is \(c_0 = r\).
Since \(q < n\), by the inductive hypothesis \(q\) has a base-\(a\) expansion
\[
q = c_1 + c_2 a + \cdots + c_m a^{m-1}.
\]
Multiplying this by \(a\) and adding \(r\) gives
\[
n = c_0 + c_1 a + c_2 a^2 + \cdots + c_m a^m,
\]
which is the desired representation.
Hence every natural number can be expressed in this way.

\smallskip
\textbf{Uniqueness.}
Suppose that \(n\) has two representations
\[
n = c_0 + c_1 a + \cdots + c_m a^m
      = d_0 + d_1 a + \cdots + d_k a^k,
\]
where \(0 \le c_i,d_j < a\).
Without loss of generality, assume \(m \le k\) and pad the shorter expansion with leading zeros if necessary.

Subtracting the two expressions yields
\[
0 = (c_0 - d_0) + (c_1 - d_1)a + \cdots + (c_k - d_k)a^k.
\]
Let \(t\) be the smallest index such that \(c_t \ne d_t\).
Then the right-hand side can be written as
\[
a^t \bigl[(c_t - d_t) + a(\text{integer})\bigr].
\]
Since \(a^t\) divides the whole expression, we obtain that \(a\) divides \((c_t - d_t)\).
But \(|c_t - d_t| < a\), hence \(c_t - d_t = 0\), a contradiction.
Therefore all digits coincide and the representation is unique.

\smallskip
\textbf{Conclusion.}
Every natural number \(n\) has a unique base-\(a\) expansion
\[
n = c_0 + c_1 a + \cdots + c_m a^m,
\qquad 0 \le c_i < a.
\]
\hfill\(\Box\)

\end{document}
